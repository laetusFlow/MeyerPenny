\chapter{Einleitung}
Hier könnte eine Einleitung stehen

\chapter{Quantencomputing als Konzept}
Quantencomputing ist, grundsätzlich gesehen, Informationsverarbeitung mittels quantenmechanischer Systeme \cite[NielsenChuang]{1}.

\section{Unterschied Bit -- Qubit}

\section{Aufgabe 2}
\section{Aufgabe 3}
\chapter{Das Meyer-Penny-Spiel}
Was ist damit gemeint?

\section{Code}
\begin{lstlisting}[caption=Python-Code \texttt{./mybestpython.py}, label=pythoncodebeispiel, language=python]
print("Hi Mom") 
# I really dig your mom
\end{lstlisting}
Ich glaube, wir brauchen ein Zitat\cite{NielsenChuang}.

\section{Programmergebnis}
Hier steht was rauskommt

\chapter{Schluss}
Sind diese Quanten eigentlich radioaktiv?
\chapter{(Mitschrieb in Vorlesung)}
\begin{itemize}
\item Zustände: Superposition von Zuständen
\item Für mehrere Zustände dind die Zustände diskret
\item Eine Messung liefert genau einen der möglichen Zustände
\item Mehrere Messungen liefern unterschiedliche konkrete Zustände. Um den Zustand $\ket{i}$ zu messen: $\left\lvert \alpha_i \right\rvert ^2 \rightarrow \sum_{i = 0}^{n} \left\lvert \alpha_i \right\rvert ^2 = 1$
\item Man kann die realen Messungen auf einem klassischen Recher mit Zufallszahlen simulieren.
$Z=\alpha\ket{0}+\beta\ket{1}$ mit $\alpha + \beta = 1$
\item Zustände werden bei der Messung zerstört.
\item Rechnen mit QBits: \\
$$
\ket{\varPsi} = \alpha\ket{0}\beta\ket{1}
$$
Operation O:
$$
O \ket{\varPsi} = \alpha'\ket{0}\beta'\ket{1}
$$
$\ket{0},\ket{1}$ sind abstrakte Symbole:
$$
\ket{0}\rightarrow \begin{pmatrix}
    1 \\ 0
\end{pmatrix}
$$
\end{itemize}
