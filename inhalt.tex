\renewcommand*{\arraystretch}{1.0} % Matrix vertical Spacing

\chapter{Einleitung}
Hier könnte eine Einleitung stehen

\chapter{Quantencomputing als Konzept}
Quantencomputing ist, grundsätzlich gesehen, Informationsverarbeitung mittels quantenmechanischer Systeme \cite[1]{NielsenChuang}.
Diese Technologie ermöglicht es, gewisse Berechnungen vielfach schneller und effizienter als auf herkömmlichen, \textit{klassischen} Computern vorzunehmen. Dabei wird ein grundlegendes Konzept der Quantenmechanik nutzbar gemacht, die \textit{Superposition} -- der Umstand, dass in einem Quantensystem ein Zustand erst im Rahmen der Beobachtung determinitert wird, davor jedoch verschiedene Zustände sich zugleich überlagern können. 


\section{Unterschied Bit -- Qbit}
Dieses Konzept der \textit{Superposition} wird im Quantencomputing dergestalt angewandt, dass die grundlegende atomare Einheit des Quantencomputings nicht, wie im Digitalrechner, das Bit ist, sondern das sogenannte \textit{QBit} (Quantenbit).
Auch dies ist ein binäres System, jedoch jeweils erst ab der Messung.
Vor der Messung kann ein Qbit eine unendliche Zahl an Zuständen zwischen den möglichen Messzuständen $0$ und $1$ einnehmen.
Mathematisch werden diese \glqq Möglichkeiten\grqq{} durch Vorfaktoren (üblicherweise $\alpha$ und $\beta$) ausgedrückt:
\begin{align}
&\alpha \cdot \ket{0}+\beta \cdot \ket{1} \label{qbitdef} \\
&\text{wobei}\notag \\ 
&\left\lvert \alpha \right\rvert ^2 + \left\lvert \beta \right\rvert ^2 = 1 & \{ \alpha, \beta \} \in \mathbb{C}
\end{align}
$\alpha$ und $\beta$ werden hierbei als \textit{Amplituden} des QBits (\ref{qbitdef}) bezeichnet.
Die Amplituden bilden gemeinsam mit der Superposition $\ket{0}+\ket{1}$ einen Vektorraum, der Zustandsvektor $(\alpha, \beta)$
\begin{align}
    \begin{pmatrix}
        \alpha\\
        \beta
    \end{pmatrix}.
\end{align}
ist die Linearkombination der zweidimensionalen Standardbasisvektoren\cite[22]{Homeister}:

\begin{align}
    \left(\begin{array}{c}
        \alpha\\
        \beta
    \end{array}\right) 
    =
    \alpha \left(\begin{array}{c}
        1\\
        0   
    \end{array}\right) 
    +
    \beta \left(\begin{array}{c}
        0\\
        1   
    \end{array}\right) 
    =
    \alpha \cdot \ket{0}+\beta \cdot \ket{1}.
\end{align}



\section{Aufgabe 2}
\section{Aufgabe 3}
\chapter{Das Meyer-Penny-Spiel}
Was ist damit gemeint?

\section{Code}
\begin{lstlisting}[caption=Python-Code \texttt{./mybestpython.py}, label=pythoncodebeispiel, language=python]
print("Hi Mom") 
# I really dig your mom
\end{lstlisting}
Ich glaube, wir brauchen ein Zitat\cite{NielsenChuang}.

\section{Programmergebnis}
Hier steht was rauskommt

\chapter{Schluss}
Sind diese Quanten eigentlich radioaktiv?
\chapter{(Mitschrieb in Vorlesung)}
\begin{itemize}
\item Zustände: Superposition von Zuständen
\item Für mehrere Zustände dind die Zustände diskret
\item Eine Messung liefert genau einen der möglichen Zustände
\item Mehrere Messungen liefern unterschiedliche konkrete Zustände. Um den Zustand $\ket{i}$ zu messen: $\left\lvert \alpha_i \right\rvert ^2 \rightarrow \sum_{i = 0}^{n} \left\lvert \alpha_i \right\rvert ^2 = 1$
\item Man kann die realen Messungen auf einem klassischen Recher mit Zufallszahlen simulieren.
$Z=\alpha\ket{0}+\beta\ket{1}$ mit $\alpha + \beta = 1$
\item Zustände werden bei der Messung zerstört.
\item Rechnen mit QBits: \\
$$
\ket{\varPsi} = \alpha\ket{0}\beta\ket{1}
$$
Operation O:
$$
O \ket{\varPsi} = \alpha'\ket{0}\beta'\ket{1}
$$
$\ket{0},\ket{1}$ sind abstrakte Symbole:
$$
\ket{0}\rightarrow \begin{pmatrix}
    1 \\ 0
\end{pmatrix}
$$
\end{itemize}
